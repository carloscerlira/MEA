\documentclass{article}
\usepackage[utf8]{inputenc}
\textheight = 25cm 
\textwidth = 15cm
\topmargin = -2.5cm 
\oddsidemargin = 1.5cm
\usepackage{amsmath}
\usepackage{mathtools, xparse}
\usepackage[shortlabels]{enumitem}
\usepackage[most]{tcolorbox}
\usepackage{bm} 

\DeclarePairedDelimiter{\norm}{\lVert}{\rVert}

\title{Tarea 4 Mecánica Analítica}
\author{Cerritos Lira Carlos}
\date{2 de Marzo del 2020}

\begin{document}
\maketitle
\section*{Problemas}
\subsection*{1.-}
\subsubsection*{2.11)}
Obtener el radio de curvatura de una curva plana en:
\begin{enumerate}[a)]
    \item Coordenadas rectangulares
    \item Coordenadas polares
\end{enumerate}
\begin{tcolorbox}
\subsubsection*{a)}
Nuestra curva esta parametrizada por:
\[ \bm{\gamma}(x) 
= x \bm{i} +  f(x) \bm{j} \]
haciendo cuentas:
\begin{align*}
    \bm{\gamma}'(x) &= \bm{i} + f'(x)\bm{j} \\
    \bm{\gamma}''(x) &= f''(x)\bm{j} \\
    \bm{\gamma}'(x) \times \bm{\gamma}''(x)
    &= f''(x)\bm{k}
\end{align*}
de donde obtenemos el radio de curvatura en el punto $\bm{\gamma}(x)$ es:
\begin{align*}
    \rho(\phi)
    &= \frac
    {\norm{\bm{\gamma}(x)}^3}
    {\norm{\bm{\gamma}'(x) \times \bm{\gamma}''(x)}} \\
    &= \frac
    {|1+f'(x)^2|^\frac{3}{2}}
    {|f''(x)|}
\end{align*}

\subsubsection*{b)}
Nuestra curva esta parametrizada por:
\[ \bm{\gamma}(\phi) = r(\phi)\bm{e_\phi} \]
haciendo cuentas:
\begin{align*}
    \bm{\gamma}'(\phi) 
    &= r'(\phi)\bm{e_\phi} + r(\phi)\bm{e_\theta} \\
    \bm{\gamma}''(\phi)
    &= r''(\phi)\bm{e_\phi} + r'(\phi)\bm{e_\theta}
    + r'(\phi)\bm{e_\theta} - r(\phi)\bm{e_\phi} \\
    \bm{\gamma}'(\phi) \times \bm{\gamma}''(\phi)
    &= 2r'(\phi)^2 \bm{k} 
    - r(\phi)r''(\phi) \bm{k} + r(\phi)^2 \bm{k}
\end{align*}
de donde obtenemos el radio de curvatura en el punto $\bm{\gamma}(\phi)$ es:
\begin{align*}
    \rho(\phi)
    &= \frac
    {\norm{\bm{\gamma}(\phi)}^3}
    {\norm{\bm{\gamma}'(\phi) \times \bm{\gamma}''(\phi)}} \\
    &= \frac
    {(r(\phi)^2 + r'(\phi)^2)^\frac{3}{2}}
    {| 2r'(\phi)^2 + r(\phi)^2 - r(\phi)r''(\phi)|}
\end{align*}
\end{tcolorbox}

\subsection*{2.-}
\subsubsection*{2.12)}
Obtener las componentes tangencial y normal de las velocidades y las aceleraciones
de las partículas de los problemas $2.1 b)$
\begin{tcolorbox}
    La posición, velocidad y aceleración:
    \begin{align*}
        \bm{r}(t) &=  3t\bm{i} - 4t\bm{j} + (t^2+3)\bm{k} \\
        \bm{v}(t) &= 3\bm{i} - 4\bm{j} + 2t\bm{k} \\
        \bm{a}(t) &= 2\bm{k}
    \end{align*}
    Para la velocidad la componente tangencial y perpendicular a la curva es:
    \begin{align*}
        \bm{v_\parallel}(t) &= 0 \\ 
        \bm{v_\bot}(t) &= \bm{v}(t)
    \end{align*}
    Para la aceleración la componente tangencial y perpendicular a la curva es:
    \begin{align*}
        \bm{a_\parallel}(t) 
        &= (\bm{u_t}(t) \cdot \bm{a}(t)) \bm{u_t}(t) \\
        &= \frac{1}{\norm{\bm{v}(t)}^2} (\bm{v}(t) \cdot \bm{a}(t))\bm{v}(t) \\
        &= \frac{4t}{\norm{\bm{v}(t)}^2}\bm{v}(t) \\
        \bm{a_\bot}(t) 
        &= \bm{a}(t) - \bm{a_\parallel}(t) \\
        &= \bm{a}(t) - \frac{4t}{\norm{\bm{v}(t)}^2}\bm{v}(t)
    \end{align*}       
    \end{tcolorbox}

\subsection*{3.-}
\subsubsection*{2.13)}
Obtener el radio de curvatura de las curvas de los problemas $2.1 b)$ para el punto en que está situado la partícula en el instante $t$.
\begin{tcolorbox}
Nuestra curva esta parametrizada por:
\[ \bm{r}(t) = 3t\bm{i} - 4t\bm{j} + (t^2 + 3)\bm{k}\] 
haciendo cuentas:
\begin{align*}
    \bm{r'}(t) &= 3\bm{i} - 4\bm{j} + 2t\bm{k} \\
    \bm{r''}(t) &= 2\bm{k} \\
    \bm{r'}(t) \times \bm{r''}(t) &= -6\bm{j} - 8\bm{i}
\end{align*}
de donde obtenemos el radio de curvatura en el punto $\bm{r}(t)$ es:
\begin{align*}
    \rho(t) &= \frac
    {\norm{\bm{r}(t)}^3}
    {\norm{\bm{r'}(t) \times \bm{r''}(t)}} \\
    &= \frac
    {(9t^2 + 16t^3 + (t^3+3)^2)^\frac{3}{2}}
    {|36 + 64|^\frac{1}{2}} \\
    &= \frac
    {(9t^2 + 16t^3 + (t^3+3)^2)^\frac{3}{2}}
    {10}
\end{align*}
\end{tcolorbox}

\subsection*{4.-}
\subsubsection*{2.15)}
En el punto $(2,1,1)$, obtener el vector unidad tangente a la intersección de la 
superficie del problema $2.14$ y a la superficie:
\[ \phi_2(x,y,z) = 3x^2-xy+y^2=11\]
\begin{tcolorbox}
    El vector tangente a la intersección de superficies esta dado por:
    \[ \bm{e_t}(\bm{x}) = \nabla \phi_1 (\bm{x}) \times \nabla \phi_2 (\bm{x}) \]
    haciendo cuentas obtenemos:
    \begin{align*}
        \nabla \phi_1 (\bm{x})
        &= (2x+2y)\bm{i} + (2x-2y+z)\bm{j} + (y+2z)\bm{k}\\
        \nabla \phi_2 (\bm{x})
        &= (6x-y)\bm{i} + (-x+2y)\bm{j}\\
        \bm{e_t}(2,2,1)
        &=\nabla \phi_1(2,2,1) \times \nabla \phi_2(2,2,1) \\
        &= (6\bm{i} + 3\bm{j} + 3\bm{z}) \times 11\bm{i} \\
        &= 33\bm{j} -33\bm{k}
    \end{align*}
    finalmente nuestro vector unitario es:
    \begin{align*}
        \bm{u_t} 
        &= \frac{\bm{e_t}}{\norm{\bm{e_t}}} \\
        &= \frac{1}{33\sqrt{2}} (33\bm{j} -33\bm{k}) \\
        &= \frac{1}{\sqrt{2}} (\bm{j} - \bm{k})
    \end{align*}
\end{tcolorbox}
\subsection*{5.-}
\subsubsection*{3.1)}
Un helicóptero aterriza con un viento cruzando en un barco en movimiento, desde el cual se observa que desciende verticalmente a 10 nudos. Si el barco tiene una velocidad de avance de 20 nudos y el viendo cruzado está soplando perpendicularmente al curso del barco a 20 nudos, encontrar la velocidad del helicóptero a través del aire.
\begin{tcolorbox}
Consideremos $S$ el sistema de un observador en reposo y $S'$ el sistema de un observador en el barco, la posición del helicóptero en el sistema $S$ está dada por:
\[ \bm{r} = \bm{R} + \bm{r'} \]
de donde obtenemos:
\begin{align*}
    \bm{v} 
    &= \bm{V} + \bm{v'} \\
    &= (20\bm{i} - 20\bm{j}) +(-10\bm{k}) \\
    &= 20\bm{i} - 20\bm{j} - 10\bm{k}
\end{align*}
donde la velocidad es:
\begin{align*}
    v &= \norm{\bm{v}} \\
    v &= (20^2 + 20^2 + 10^2)^\frac{1}{2} \\
    v &= 30 nudos    
\end{align*}
\end{tcolorbox}


\end{document}






































 