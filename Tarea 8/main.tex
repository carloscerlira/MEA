\documentclass{article}
\usepackage[utf8]{inputenc}
\textheight = 25cm 
\textwidth = 15cm
\topmargin = -2.5cm 
\oddsidemargin = 1.5cm
\usepackage{float}
\usepackage{graphicx}
\graphicspath{{./images/}}

\usepackage{amsmath}
\usepackage{mathtools, xparse}
\usepackage[shortlabels]{enumitem}
\usepackage[most]{tcolorbox}
\usepackage{adjustbox}
\usepackage{bm} 

\DeclarePairedDelimiter{\norm}{\lVert}{\rVert}

\title{Tarea 8 Mecánica Analítica}
\author{Cerritos Lira Carlos}
\date{6 de Mayo del 2020}

\begin{document}
\maketitle
\section*{Problemas}
\section*{1.-}
\subsection*{6.7)}
En un espectrómetro de masas, se acelera un ion posotivo de una sola carga 
($q=1.602 \times 10^{-19} coloumbs$) por medio de una diferencia de potencial de 
$1000 voltios$. Luego pasa por un campo magnético uniforme en el que $B=0.1weber/m^2$,
y se desvía en una trayectoria circular de $0.182m$ de radio. Determinar:
\begin{enumerate}[a)]
    \item La velocidad del ion.
    \item La masa del ion en kilogramos y unidades de masa atómica.
    \item El número de masa del ion.
\end{enumerate}
\begin{tcolorbox}[breakable]
    \subsubsection*{a)}
    Despejamos $v^2$ de la ecuación del trabajo y la energía:
    \[ v^2 = \frac{2qV}{m} \]
    encontramos la nuevamente la velocidad usando el movmiento circular que sigue una vez entra al campo magnético, donde:
    \[ v = \frac{rqB}{m} \]
    juntando ambas obtenemos:
    \[ v = \frac{2V}{rB} = 1.1 \times 10^5 m/seg\]
    \subsubsection*{b)}
    De la equación del trabajo y la energía despejamos $m$
    \[ m = \frac{2qV}{v^2} = 2.64 \times 10^{-26}kg = 15.94 uma\]
    \subsubsection*{c)}
    \[ A = 15 \]
\end{tcolorbox}

\section*{2.-}
\subsection*{6.9}
En la posición $x=0, y=0$, un cañón tiene un alcance máximo $l_m$. Determinar los dos 
ángulos de elevación para hacer blanco en el punto:
\[ x = l_m/2, \quad y=l_m/4 \]
\begin{tcolorbox}[breakable]
    Usamos las coordenadas $x,y$, donde tomamos un sistema de referencia
    tal que al incio ambas sean cero:
    \begin{align*}
        T &= \frac{m}{2}(\dot{x}^2+\dot{y}^2) \\
        U &= mgy 
    \end{align*}
    usando Lagrange llegamos a las ecuaciones:
    \begin{align*}
        m\ddot{x} &= 0 \\
        m\ddot{y} + mg &= 0
    \end{align*}
    de donde obtenemos:
    \begin{align*}
        x &= v_{0x}t \\
        y &= -\frac{gt^2}{2}-v_{0yt}
    \end{align*}
    de la ecuación para $y$ despejamos el tiempo en función de $\theta$:
    \[ t = \frac{2v_0sin\theta}{g} \]
    la distancia recorrida para este tiempo es:
    \begin{align*}
        x 
        &= v_{0x}t \\
        &= v_0cos\theta \frac{2v_0sin\theta}{g} \\
        &= \frac{2v_0^2}{g}sin\theta cos\theta
    \end{align*}
    de donde obtenemos el alcance máximo se obtiene cuando $\theta=\frac{\pi}{4}$, dando:
    \[ l_m = \frac{v_0^2}{g} \]
    veremos el tiempo que tarda el proyectil en alcanzar $l_m/2$
    \[ t = \frac{l_m}{2v_0cos\theta} \]
    para este tiempo queremos $y = \frac{l_m}{4}$, sustituyendo en la ecuación de movimiento:
    \begin{align*}
        \frac{l_m}{4} &= -\frac{g}{2}\frac{l_m^2}{4v_0^2cos^2\theta} + v_0sin\theta\frac{l_m}{2v_0cos\theta} \\
        \frac{1}{4} &= -\frac{gl_m}{8v_0^2cos^2\theta} + \frac{tan\theta}{2} \\
        \frac{1}{4} &= -\frac{1}{8}(1+tan^2\theta) + \frac{1}{2}tan\theta \\
        0 &= tan^2\theta - 4tan\theta + 3
    \end{align*}
    cambiando nuestra variable por $x=tan\theta$, encontramos las soluciones:
    \begin{align*}
        &x_1 = 1, \quad x_2 = 3 \\
        \theta_1 &= 45, \quad \theta_2 = 71.5
    \end{align*}
\end{tcolorbox}

\section*{3.-}
\subsection*{8.13}
Una cuenta de masa $3m$ puede deslizarse horizontalmente sin rozamiento por un alambre 
como se indica en la figura $8-14$. Unido a la cuerta hay un péndulo doble. Si, en una 
posición cercana a la de su equilibrio, se deja el sistema en libertad, a partir del
reposo, las masas oscilan en el plano de la figura a un lado y al otro de la vertical.
\begin{enumerate}[a)]
    \item Escriba las ecuaciones de Lagrange del movimiento del sistema.
    \item Hállese las aceleraciones cuando los desplazamientos y las aceleraciones son pequeñas.
\end{enumerate}
\begin{figure}[H]
    \centering
    \includegraphics[scale=0.8]{p3_pendulum.png}
\end{figure}
\begin{tcolorbox}[breakable]
    \subsubsection*{a)}
    Sean $x,x_1,x_2$ la posición de las masas, definimos los vectores:
    \begin{align*}
        \bm{e}_{r_1} &= sin\theta_1\bm{i} - cos\theta_1\bm{j} \\
        \bm{e}_{\theta_1} &= cos\theta_1\bm{i} + sin\theta_1\bm{j} 
    \end{align*}
    con ayuda de estos vectores describimos el movimiento:
    \begin{align*}
        \bm{x} &= x\bm{i} \\
        \bm{x}_1 &= \bm{x} + l_1\bm{e}_{r_1} \\
        \bm{x}_2 &= \bm{x}_1 + l_2\bm{e}_{r_2} 
    \end{align*}
    de donde obtenemos:
    \begin{align*}
        \bm{v}
        &= \dot{x}\bm{i} \\
        \bm{v}_1 
        &= \dot{x}\bm{i} + l_1\dot{\theta}_1\bm{e}_{\theta_1} \\
        &= (\dot{x} + l_1\dot{\theta}_1cos\theta_1 )\bm{i} - l_1\dot{\theta}_1sin\theta_1 \bm{j} \\
        \bm{v}_2 
        &= \dot{x}\bm{i} + l_1\dot{\theta}_1\bm{e}_{\theta_1} + l_2\dot{\theta}_2\bm{e}_{\theta_2} \\
        &= (\dot{x} + l_1\dot{\theta}_1cos\theta_1 + l_2\dot{\theta}_2cos\theta_2)\bm{i} - (l_1\dot{\theta}_1sin\theta_1 + l_2\dot{\theta}_2sin\theta_2) \bm{j} 
    \end{align*}
    calculamos la energía cinética y potencial utilizando las coordenadas generalizadas $\theta_1,\theta_2$:
    \begin{align*}
        T 
        &= \frac{3}{2}m\dot{x}^2 + \frac{1}{2}ml_1^2\dot{\theta}_1^2sin^2\theta_1 + \frac{1}{2}m(l_1 \dot{\theta}_1 cos\theta_1 + \dot{x})^2 \\
        &+ \frac{1}{2}m(l_1\dot{\theta}_1 sin\theta_1 + l_2\dot{\theta}_2 sin\theta_2)^2 \\
        &+ \frac{1}{2}m(l_1\dot{\theta}_1cos\theta_1 + l_2\dot{\theta}_2 cos\theta_2 + \dot{x})^2 \\
        U
        &= mgy_1 + mgy_2 \\
        &= -mgl_1cos\theta_1 - mg(l_1cos\theta_1 + l_2cos\theta_2) \\
        &= -2mgl_1cos\theta_1 - mgl_2cos\theta_2
    \end{align*}
    Obtenemos las ecuaciones de movmiento utilizando la relación:
    \begin{align*}
        \frac{\partial L}{\partial q_i} - \frac{d}{dt}\frac{\partial L }{\partial \dot{q}_i} &= 0 
    \end{align*}
    para $\theta_1$ tenemos:
    \begin{align*}
        &\quad ml_1^2\dot{\theta}_1^2sin\theta_1 cos\theta_1 
        - m(l_1\dot{\theta}_1)cos\theta_1 + \dot{x})(l_1\dot{\theta}_1sin\theta_1) \\
        &+m(l_1\dot{\theta}_1 sin\theta_1 + l_2\dot{\theta}_2 sin\theta_2)l_1\dot{\theta}_1cos\theta_1 \\
        &+m(l_1\dot{\theta}_1cos\theta_1 + l_2\dot{\theta}_2 cos\theta_2 + \dot{x})l_1\dot{\theta}_1sin\theta_1 - 2mgl_1sin\theta_1 \\
        &-\frac{d}{dt}\big(
        ml_1^2\dot{\theta}_1sin^2 \theta_1 + m(l_1\dot{\theta}_1cos\theta_1 + \dot{x})l_1cos\theta_1 \\
        &+m(l_1\dot{\theta}_1 sin\theta_1 + l_2\dot{\theta}_2 sin\theta_2)l_1sin\theta_1 \\
        &+m(l_1\dot{\theta}_1cos\theta_1 + l_2\dot{\theta}_2 cos\theta_2 + \dot{x})l_1cos\theta_1 \big)
        = 0
    \end{align*}
    de forma similar se encuentra una ecuación para $\theta_2$.
    \subsubsection*{b)}
    Usaremos la aproximación $cosx = 1$, $sinx=x$, además despreciaremos términos pequeños,
    calculamos la energía potencial y cinética:
    \begin{align*}
        T
        &= \frac{3}{2}m\dot{x}^2 + \frac{1}{2}m(l_1\dot{\theta}_1 + \dot{x})^2 + \frac{1}{2}m(l_1\dot{\theta}_1+l_2\dot{\theta}_2 + \dot{x})^2 \\
        U
        &= -2mgl_1(1-\tfrac{1}{2}\theta_1^2) - mgl_2(1-\tfrac{1}{2}\theta_2^2) \\
        &= -mg(2l_1+l_2) + mgl_1\theta_1^2 + \frac{1}{2}mgl_2\theta_2^2 \\
        U
        &= mgl_1\theta_1^2 + \frac{1}{2}mgl_2\theta_2^2
    \end{align*}
    donde redefinimos $U$ sabiendo que podemos quitar constantes, usando nuevamente las ecuaciones de Lagrange para $\theta_1$ obtenemos:
    \begin{align*}
        -2mgl_1\theta_1 - \frac{d}{dt} \left(m(l_1\dot{\theta}_1+\dot{x})l_1 + m(l_1\dot{\theta}_1 + l_2\dot{\theta}_2 + \dot{x})l_1 \right) &= 0
    \end{align*}
    de forma similar se encuentr auna ecuación para $\theta_2$

\end{tcolorbox}

\section*{4.-}
\subsection*{Demostración 8.15}
\[ [x_i,l_j] = \sum_{k}e_{ijk}x_k \]
\begin{tcolorbox}[breakable]
    Usaremos las propiedades:
    \[ [x_i,x_j] = 0, \quad [x_i,p_j] = \delta_{ij}, \quad [a,b] = -[b,a] \]
    para $i=1$ hacemos cálculos:
    \begin{align*}
        [x_1,l_1]
        &= [x_1, x_2p_3-x_3p_2] \\
        &= [x_1,x_2p_3] - [x_1,x_3p_2] \\
        &= p_3[x_1,x_2] + x_2[x_1,p_3] - x_3[x_1,p_2] + p_2[x_1,x_3] \\
        &= x_2[x_1,p_3] - x_3[x_1,p_2] \\ 
        &= 0 \\
        [x_1,l_2]
        &= [x_1,x_3p_1-x_1p_3] \\
        &= x_3[x_1,p_1] - x_1[x_1,p_3] \\
        &= x_3 \\
        [x_1,l_3] 
        &= [x_1, x_1p_2-x_2p_1] \\
        &= x_1[x_1,p_2]-x_2[x_1,p_1] \\
        &= -x_2
    \end{align*}
    haciendo permutaciones ciclicas y cambiando signo cuando no se siga el orden correcto obtenemos:
    \begin{align*}
        [x_2,l_1]
        &= -x_3 \\
        [x_2,l_2] 
        &= 0 \\
        [x_2,l_3]
        &= x_1 \\
        [x_3,l_1] 
        &= -x_2 \\
        [x_3,l_2]
        &= x_1 \\
        [x_3,l_3]
        &= 0
    \end{align*}
    donde se comprueba que en efecto:
    \[ [x_i,l_j] = \sum_{k}e_{ijk}x_k \]
\end{tcolorbox}

\section*{5.-}
\subsection*{Demostración 8.18}
\[ \frac{\partial }{\partial x}[X,Y] = [\frac{\partial X}{\partial x},Y] + [X,\frac{\partial Y}{\partial x}] \]
\begin{tcolorbox}[breakable]
    \begin{align*}
        \frac{\partial }{\partial x}[X,Y]
        &=\frac{\partial }{\partial x} \sum_{i} \left(
        \frac{\partial X}{\partial q_i} \frac{\partial Y}{\partial p_i}
        -\frac{\partial Y}{\partial p_i}\frac{\partial X}{\partial q_i} \right) \\
        &= \sum_{i} \left(
        \frac{\partial}{\partial x}\frac{\partial X}{\partial q_i} \frac{\partial Y}{\partial p_i}
        +\frac{\partial X}{\partial q_i}\frac{\partial}{\partial x}  \frac{\partial Y}{\partial p_i} 
        -\frac{\partial}{\partial x} \frac{\partial Y}{\partial p_i}\frac{\partial X}{\partial q_i} 
        -\frac{\partial Y}{\partial p_i}\frac{\partial}{\partial x} \frac{\partial X}{\partial q_i}
        \right) \\
        &= \sum_{i} \left(
        \frac{\partial}{\partial q_i}\frac{\partial X}{\partial x} \frac{\partial Y}{\partial p_i}
        -\frac{\partial Y}{\partial p_i}\frac{\partial}{\partial q_i} \frac{\partial X}{\partial x}
        \right)
        +\sum_{i} \left( 
        +\frac{\partial X}{\partial q_i}\frac{\partial}{\partial p_i} \frac{\partial Y}{\partial x} 
        -\frac{\partial}{\partial p_i} \frac{\partial Y}{\partial x}\frac{\partial X}{\partial q_i} 
        \right) \\
        &= [\frac{\partial X}{\partial x},Y] + [X,\frac{\partial Y}{\partial x}] 
    \end{align*}
\end{tcolorbox}

\end{document}