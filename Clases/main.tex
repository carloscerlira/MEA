\documentclass{article}
\usepackage[utf8]{inputenc}
\textheight = 25cm 
\textwidth = 15cm
\topmargin = -2.5cm 
\oddsidemargin = 1.5cm
\usepackage{float}
\usepackage{graphicx}
\graphicspath{{./images/}}

\usepackage{amsmath}
\usepackage{mathtools, xparse}
\usepackage[shortlabels]{enumitem}
\usepackage[most]{tcolorbox}
\usepackage{adjustbox}
\usepackage{bm} 

\DeclarePairedDelimiter{\norm}{\lVert}{\rVert}

\title{Tarea 6 Mecánica Analítica}
\author{Cerritos Lira Carlos}
\date{28 de Marzo del 2020}

\begin{document}
\section*{4/13/20}
\subsection*{Pendulo esférico}
Definimos la Lagrangiana:
\[L = T-U \]
si solo hay fuerzas conservativas, se satisface:
\[ \frac{d}{dt}\frac{\partial L}{\partial \dot{q}_i} - \frac{\partial L}{\partial q_i} = 0\]
\begin{tcolorbox}
    \subsubsection*{5-5)}
    El punto de sporte de un péndulo simple se mueve en una circunferencia
    vertical de radio $R$ con una velocidad constante $v$. Hallar la ecuación 
    de movimiento de Lagrangiana para el péndulo si:
    \[ U = mgRsin\phi -mlgcos\theta \].
    Se tienen lo siguiente:
    \begin{align*}
        x &= Rcos\phi - lsin\theta \\
        y &= Rsin\phi - lcos\theta \\ 
        T 
        &= \frac{m}{2} (\dot{x}^2 + \dot{y}^2) \\
        &= \frac{m}{2} ( R^2 \dot{\phi}^2 + l^2 \dot{\theta}^2 + 2rl\dot{\theta}\dot{\phi}sin\theta)
    \end{align*}
    \subsubsection*{5-7)}
    Se tiene lo siguiente:
    \begin{align*}
        L 
        &= T-U \\
        &= \frac{m}{2} (\dot{\rho}^2 + \rho^2 \dot{\varphi}^2 + \dot{z}^2)-mg\rho cos\varphi
    \end{align*}
    Calculamos:
    \begin{align*}
        \frac{d}{dt}\frac{\partial L}{\partial \dot{q}_i} - \frac{\partial L}{\partial q_i} &= 0 \\
        \lambda &= \frac{3mgacos\varphi - 2E}{a}
    \end{align*}
    se separa cuando $\lambda = 0$, donde $h= acos\varphi = \frac{2}{3}a$
\end{tcolorbox}
\subsection*{Fuerza conservativa}
Trabajo realizado solo depende de los puntos de inicio y final.
Consideremos la siguiente fuerza:
\[ F(x,y,z) = xy\hat{i} + 2z\hat{j} - (2x^2-y^2)\hat{k} \]
trayectoria $x=y=z$, $x=y^2=z$. \\
Observamos que el trabajo realizado es diferente, dada una fuerza queremos saber 
si es conservativa. 
\end{document}