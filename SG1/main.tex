\documentclass{article}
\usepackage[utf8]{inputenc}
\textheight = 25cm 
\textwidth = 15cm
\topmargin = -2.5cm 
\oddsidemargin = 1.5cm
\usepackage{amsmath}
\usepackage{mathtools, xparse}
\usepackage[shortlabels]{enumitem}
\usepackage[most]{tcolorbox}
\usepackage{bm}


\DeclarePairedDelimiter{\norm}{\lVert}{\rVert}

\title{Guia de estudio parcial 1}
\author{Cerritos Lira Carlos}
\date{2 de Marzo del 2020}

\begin{document}
\maketitle
\subsection*{Ecuaciones de movimiento de Lagrange}
\subsubsection*{Fuerzas generalizadas}
Dadas las funciones:
\begin{align*}
    \bm{b}_i &= \frac{d\bm{r}}{dq_i} \\
    q_i &= q_i(x,y,z) \\
    \bm{v}_i &= \nabla q_i
\end{align*}
escribimos al vector fuerza como:
\begin{align*}
    Q_i &= \bm{F} \cdot \bm{b}_i \\
    \bm{F} &= \sum_{i=1}^3 Q_i\bm{v}_i
\end{align*}
se encuentra que:
\begin{align*}
    T &= \frac{1}{2} \sum_{i=1}^3 p_i\dot{q}_i \\
    \bm{F} \cdot \bm{b}_i &= \frac{d}{dt}\frac{\partial T}{\partial \dot{q_i}} - \frac{\partial T}{\partial q_i}
\end{align*}
\begin{tcolorbox}[breakable]
    \subsubsection*{Partícula en un plano bajo la acción de una fuerza central}
    Consideremos una fuerza de la forma:
    \begin{align*}
        \bm{F} &= -mw_0^2\bm{r}
    \end{align*}
\end{tcolorbox}

\end{document}