\documentclass{article}
\usepackage[utf8]{inputenc}
\textheight = 25cm 
\textwidth = 15cm
\topmargin = -2.5cm 
\oddsidemargin = 1.5cm
\usepackage{amsmath}
\usepackage{mathtools, xparse}
\usepackage[shortlabels]{enumitem}
\usepackage[most]{tcolorbox}
\usepackage{bm}


\DeclarePairedDelimiter{\norm}{\lVert}{\rVert}

\title{Guia de estudio parcial 1}
\author{Cerritos Lira Carlos}
\date{2 de Marzo del 2020}

\begin{document}
\maketitle
\section*{Tensors}
Let $V$ be a vector space of dimension $n$ over a field $F$, $\bm{x} \in V$, 
let $\bm{e_1,e_2,..,e_n}$ be a basis for $V$ then:
\[ x = x_i \bm{e_i} \]
Let $\bm{\tilde{e_1}, \tilde{e_2},...,\tilde{e_3}}$ be another basis for $V$, then:
\[ x = \tilde{x}_i \bm{\tilde{e_i}} \]
where:
\begin{align*}
    \bm{\tilde{e_i}} &= S_{ij} \bm{e_j} 
\end{align*}
then we have the following relation:
\begin{align*}
    x_i &= S_{ij}\tilde{x_j} \\
    \tilde{x}_i &= T_{ij}x_j
\end{align*}
from this relation we obtain:
\begin{align*}
    \frac{\partial x_i}{\partial \tilde{x}_j} 
    &= S_{ij}
\end{align*}

\begin{tcolorbox}
    In this example $V=R^2$ and $F=R$.
    \begin{align*}
        \bm{e_1,e_2} &= \bm{i,j} \\
        \bm{\tilde{e_1}, \tilde{e_2}} &= \bm{e_r,e_\theta}      
    \end{align*}
   let $\bm{x} \in V$:
   \begin{align*}
       \bm{x} &= x\bm{i} + y\bm{j} \\
       \bm{x} &= r\bm{e_r}
   \end{align*}
   we know:
   \begin{align*}
       \bm{e_r} &= cos\theta \bm{i} + sin\theta \bm{j} \\
       \bm{e_\theta} &= -sin\theta \bm{i} + cos\theta \bm{j}
   \end{align*}
   hence we have:
   \begin{align*}
        S 
        &= \left( \begin{matrix} 
        cos\theta & -sin\theta \\ 
        sin\theta & cos\theta 
        \end{matrix} \right) 
        \\
        T
        &= \left( \begin{matrix} 
        cos\theta & sin\theta \\ 
        -sin\theta & cos\theta 
        \end{matrix} \right)
   \end{align*} 
   we know the following functions:
   \begin{align*}
        x(r,\theta)
        &= rcos\theta \\
        y(r,\theta)
        &= rsin\theta
   \end{align*}

\end{tcolorbox}


\section*{Movement in different coordinate systems}
Let $\bm{x}$ be the function that describes the position of our object 
as a function of time. 
\subsection*{Cylindrical coordinates}
\begin{align*}
    \bm{r} &= x\bm{i} + y\bm{j} + z\bm{k} \\
    \bm{r} &= \rho\bm{e_r} + z\bm{k}
\end{align*}
from where we obtain:
\begin{align*}
    \bm{\dot{r}} 
    &= \dot{\rho}\bm{e_r} + \rho \dot{\theta} \bm{e_\theta} + \dot{z}\bm{k} \\
    \bm{\ddot{r}} 
    &= (\ddot{\rho} - \rho \dot{\theta}^2)\bm{e_r}
    + (\rho \ddot{\theta} + 2\dot{\rho }\dot{\theta})\bm{e_\theta}
    + \ddot{z}\bm{k}
\end{align*}

\subsection*{Spherical coordinates}
\begin{align*}
    \bm{r}
    &= x\bm{i} + y\bm{j} + z\bm{k} \\
    \bm{r}
    &= 
\end{align*}

\section*{Generalized coordinates}
Let $\bm{r}$ be the function that returns the position of our object at a time $t$:
\begin{align*}
    \bm{r} &= x\bm{i} + y\bm{j} + z\bm{k} 
\end{align*}
define:
\begin{align*}
    \bm{b_i} 
    &= \frac{\partial \bm{r}}{\partial q_i}
    = h_i\bm{e_i} \\
    \bm{v_i} &= \nabla x_i 
\end{align*}
we want to know:
\begin{align*}
    \bm{r} &= r_i \bm{e_i} \\
    \bm{\dot{r}} &= v_i \bm{e_i} \\
    \bm{\ddot{r}} &= a_i \bm{e_i}
\end{align*}
 
\section*{Motion in other systems of reference}
\subsubsection*{Translation}
Let $\bm{r}(t)$ be the function that returns the position of an object at time t, where:
\begin{align*}
    \bm{r}(t) &= \bm{R}(t) + \bm{r'}(t)
\end{align*}
where:
\begin{align*}
    \bm{r'}(t) &= x'(t)\bm{i} + y'(t)\bm{j} + z'(t)\bm{k}
\end{align*}
then we have the relation:
\begin{align*}
    \frac{d\bm{r}}{dt}
    &= \frac{d\bm{R}}{dt} + \frac{d\bm{r'}}{dt}
\end{align*}

\subsubsection*{Rotation}
Let $\bm{r}(t)$ be the funcion that returns the position of an object at time $t$, where:
\begin{align*}
    \bm{r}(t) 
    &= x'(t)\bm{i}(t) + y'(t)\bm{j}(t) + z'(t)\bm{k}(t) \\
    \frac{d\bm{r}}{dt}(t)
    &= \frac{dx'}{dt}(t)\bm{i}(t) + \frac{dy'}{dt}(t)\bm{j}(t) + \frac{dz'}{dt}(t)\bm{k}(t)
    + \bm{\Omega}(t) \times \bm{r}(t) 
\end{align*}
let's define:
\begin{align*}
    \bm{r'}(t) &= x'(t)\bm{i} + y'(t)\bm{j} + z'(t)\bm{k}
\end{align*}
then we can write the derivative as:
\begin{align*}
    \frac{d\bm{r}}{dt}
    &= \frac{d\bm{r'}}{dt} + \bm{\Omega} \times \bm{r'} 
\end{align*}
we can find the acceleration:
\begin{align*}
    \frac{d^2\bm{r}}{dt^2} 
    &=\frac{d}{dt} \left( \frac{d\bm{r'}}{dt} + \bm{\Omega} \times \bm{r'} \right) 
    + \bm{\Omega} \times \left(\frac{d\bm{r'}}{dt} + \bm{\Omega} \times \bm{r'} \right) \\
    &=\frac{d^2\bm{r'}}{dt^2} 
    + \frac{d \bm{\Omega}}{dt} \times \bm{r'} + \bm{\Omega} \times \frac{d\bm{r'}}{dt}
    + \bm{\Omega} \times \frac{d\bm{r'}}{dt} + \bm{\Omega} \times (\bm{\Omega} \times \bm{r'}) \\
    &=\frac{d^2\bm{r'}}{dt^2}
    + \frac{d\bm{\Omega}}{dt} \times \bm{r'}
    + 2\bm{\Omega} \times \frac{d\bm{r'}}{dt}
    + \bm{\Omega} \times (\bm{\Omega} \times \bm{r'}) \\
\end{align*}

\subsubsection*{Translation plus rotation}
Let $\bm{r}(t)$ the function that returns the position of an object at time $t$ where:
\begin{align*}
    \bm{r}(t) 
    &= \bm{R}(t) + x(t)\bm{i}(t) + y(t)\bm{j}(t) + z(t)\bm{k}(t)
\end{align*}
thus we have:
\begin{align*}
    \frac{d\bm{r}}{dt} 
    &= \frac{d (\bm{R'+r'})}{dt} + \bm{\Omega} \times (\bm{R'} + \bm{r'})
\end{align*}
\begin{tcolorbox}
    Example: let's examinate a plomada, we have the conditions:
    \begin{align*}
        \bm{R}(t) &= R\bm{k}(t) \\
        \frac{d^2\bm{r}}{dt^2}(t) &= -g\bm{k}(t) 
    \end{align*}
    and we can ignore all the other terms, thus we have:
    \begin{align*}
        \frac{d^2\bm{r'}}{dt^2}
        &= -\frac{d^2\bm{r}}{dt^2} + \bm{\Omega} \times (\bm{\Omega} \times \bm{R'})
    \end{align*}
\end{tcolorbox}

\end{document}