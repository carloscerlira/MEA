\documentclass{article}
\usepackage[utf8]{inputenc}
\textheight = 25cm 
\textwidth = 15cm
\topmargin = -2.5cm 
\oddsidemargin = 1.5cm
\usepackage{float}
\usepackage{graphicx}
\graphicspath{{./images/}}

\usepackage{amsmath}
\usepackage{mathtools, xparse}
\usepackage[shortlabels]{enumitem}
\usepackage[most]{tcolorbox}
\usepackage{adjustbox}
\usepackage{bm} 

\DeclarePairedDelimiter{\norm}{\lVert}{\rVert}

\title{Tarea 5 Mecánica Analítica}
\author{Cerritos Lira Carlos}
\date{18 de Marzo del 2020}

\begin{document}
\maketitle
\section*{Problemas}
\section*{1.- }
\subsection*{3.4)}
Un niño monta un "caballito" que sube y baja sinusoidalemtne $h=h_0sin(wt)$ con 
relación a un tiovivo que gira alrededor de la vertical con una velocidad(tangencial)
constante $\Omega$. si el niño está a una distancia $c$ del eje de rotación, 
hallése una expresión de su aceleración relativa al suelo en funcion de $\Omega, c, h_0, w$ 
y $t$. 
\begin{tcolorbox}[breakable]
    Sea $r$ la función que regresa la posición para un tiempo $t$, se tiene:
    \begin{align*}
        \bm{r} &= c\bm{e_r} + h_0sin(wt)\bm{k} \\
        \bm{v} &= -c\Omega\bm{e_\theta} + h_0wcos(wt)\bm{k} \\
        \bm{a} &= -c\Omega^2\bm{e_r} - h_0w^2sin(wt)\bm{k} 
    \end{align*}
\end{tcolorbox}

\section*{2.-}
\subsection*{4.2)}
Encontrar la posición en un tiempo $t$ de una partícula de masa $m$, cuando la fuerza aplicada
es $F=2mcos(wt)$ y $x=8$ a $t=0$ y $x=-b$ a $t=\frac{\pi}{2w}$.
\begin{tcolorbox}[breakable]
    Tenemos la siguiente información:
    \begin{align*}
        \bm{a} &= 2cos(wt) \bm{i} \\
        \bm{v} &= \left(\frac{2}{w}sin(wt) + v_0 \right)\bm{i} \\ 
        \bm{x} &= \left(-\frac{2}{w^2}cos(wt) + v_0t + \frac{2}{w^2} + x_0 \right) \bm{i}
    \end{align*}
    podemos encontrar el valor de $v_0$ mediante la relación $\bm{x}(\frac{\pi}{2w}) = -b$
    de donde tenemos:
    \begin{align*}
        v_0 &= \frac{\frac{2}{w^2}cos\frac{\pi}{2} - \frac{2}{w^2} - x_0 - b}{\frac{\pi}{2}}
    \end{align*}
\end{tcolorbox}

\section*{3.-}
\subsection*{4.4)}
\begin{enumerate}[a)]
    \item Si la velocidad límite de caída de un hombre de $80kg$, con paracaídas, es 
    la misma que tendría al caer libremente $0.75m$; hallar el valor de esta velocidad 
    límite y la constante de amortiguamiento $k$ (supóngase $F_{amort} = -mkv$)
    \item Supongamos ahora que el hombre cae libremente (partiendo del reposo) durante 
    5 segundos y que después abre su paracaídas. Luego de otros 5 segundos, ¿cual será su 
    velocidad? 
\end{enumerate}
\begin{tcolorbox}[breakable]
    
\end{tcolorbox}

\section*{4.-}
\subsection*{4.7)}
Una partícula de masa $m$ tiene aplicada una fuerza $F=-kx^2$. Si $\dot{x} = v_0$ 
cuando $x=0$, hállese:
\begin{enumerate}[a)]
    \item la ecuación de la energía
    \item el punto de retorno
    \item la velocidad en cualquier posición
\end{enumerate} 
\begin{tcolorbox}[breakable]
    
\end{tcolorbox}

\section*{5.-}
\subsection*{3.3)}
Un semicilindro se balance sinusoidalmente sin deslizamiento, como se muestra en la figura 
$3-11$, de tal forma que $\theta = sin2t$. 
\begin{enumerate}[a)]
    \item Cuando pasa por la posición neutra $\theta=0$, ¿cuál es la aceleración del punto
    de contacto con la superfice fija?. 
    \item Cuando el semicilindro está al ángulo máximo de 1 radían ¿cuál es la aceleración 
    del punto de contacto con la superficie fija?
\end{enumerate}
\begin{tcolorbox}[breakable]
    
\end{tcolorbox}
\end{document}
